\documentclass[10pt,a4paper]{article}
\usepackage[utf8]{inputenc}
\usepackage[italian]{babel}
\usepackage{amsmath}
\usepackage{amsfonts}
\usepackage{amssymb}
\usepackage{graphicx}
\usepackage{fullpage}
\usepackage{hyperref}
\usepackage{graphicx}
\usepackage{caption}
\usepackage{subcaption}

\author{Michele Carignani, Alessandro Lenzi}
\title{Analisi aggregata per ore}
\begin{document}

\maketitle

\section{Generazione dei grafi orari}

Per prima cosa i dati sono stati aggregati per ora. I dati originali del dataset 
%todo sistemare dataset
\href{https://dandelion.eu/datagem/telecom-mi-to-mi/description/}{Telecommunications - MI to MI} sono nel formato:
\begin{verbatim}
timestamp \t SourceId \t DestId \t Stregth
\end{verbatim}
e sono stati suddivisi in 24 file (uno per ogni ora) e aggregati, per cui
ogni file contiene (al massimo\footnote{poichè certi nodi possono non avere chiamate in uscita
in una certa fascia oraria.}) un record per ogni nodo nel formato:
\begin{verbatim}
SourceId \t DestId:Strength [\t DestId:Strength]
\end{verbatim}

A questo punto i pesi sugli archi (sopra chiamati \verb!Strength!) sono stati riscalati rispetto alla somma
dei valori della stella uscente di un nodo, ottenendo la probabilità di transire dal nodo $i$ al 
nodo $j$, ovvero:
$$ sumStrength_i = \sum_{j \in FS(i)} Strength_{ij} $$
$$ probability_{ij} = \frac{Strength_{ij}}{sumStrength_i} $$

\section{Ricerca delle componenti fortemente connesse}

Per ricercare le componenti fortemente connesse (in seguito CFC) sono stati utilizzati sia diversiti tipi di taglio degli archi, sia diverse strategie di visita.

\subsection{Tagli}
Un taglio degli archi su un valore $x$ significa utilizzare per la visita solo gli archi con peso (ovvero valore di probabilità) maggiore o uguale a $x$. Sono stati eseguiti test sui valori 0.001, 0.005, 0.007, 0.009, 0.01, 0.05,
0.07, 0.08, 0.09 e 0.1.

Vedremo i risultati dei tagli 0.005 e 0.05.

\subsection{Strategie di visita}
\textbf{Le strategie di visita} utilizzate sono 5 e impiegano i dati del dataset 
\href{https://dandelion.eu/datagem/telecom-sms-call-internet-mi/description/}{"Telecommunications - SMS, Call, Internet - MI"}:
a partire dai record del formato
\begin{verbatim}
SquareID \t Timestamp \t .. ChiamateInUscita ..
\end{verbatim}
per ogni ora sono stati generati file con record
\begin{verbatim}
SquareID \t AggregatedCalls
\end{verbatim}
che permettono di capire quale sia il valore assoluto proporzionale a tutte le chiamate in uscita dallo square
$ID$ in una certa fascia oraria.
In questo modo è possibile iniziare la visita del grafo non da nodi ordinati lessicograficamente ma in ordine
(crescente o decrescente) di traffico in uscita.
Le strategie inoltre si differenziano per il modo di ordinare la stella uscente da un nodo.
Perciò le strategie definite sono:
\begin{itemize}
\item SCC1: visita il grafo in ordine crescente di traffico uscente,
selezionando prima gli archi con probabilità maggiore;
\item SCC2: visita i nodi per traffico decrescente e con archi selezionati per probabilità crescente;
\item SCC3: visita i nodi per traffico decrescente e gli archi per probabilità crescente;
\item SCC4: visita dei nodi per traffico crescente e archi per probabilità decrescente;
\item Stable: Esegue la ricerca delle componenti fortemente connesse selezionando i nodi in ordine crescente di traffico telefonico \textbf{giornaliero} uscente e gli archi in ordine di probabilità decrescente.
\item Stable con percentili: come sopra, ma tagliando il grafo utilizzando dei percentili.
\end{itemize}
 
\section{Risultati}

\subsection{Statistiche}

In fig. \ref{img:probs} sono mostrate le statistiche sui pesi degli archi come probabilità.
\begin{figure}
 \includegraphics[scale=.6]{img/probs15nov.png}
 \caption{Statistiche sui pesi degli archi, 15 novembre. Sulle ascisse le fascie orarie, sulle ordinate le probabilità.}
 \label{img:probs}
\end{figure}

\subsection{Componenti Fortemente Connesse}

Le CFC sono state disegnate su mappe (utilizzando \verb!gnuplot!) assegnando una funzione $z$ alle celle di
coordinate $(x,y)$ così definita:
$$
z(x,y) =
\begin{cases}
0, & \text{se $(x,y)$ non appartiene ad alcuna CFC,} \\
10k, & \text{se $(x,y)$ appartiene alla k-esima CFC trovata.}
\end{cases}
$$

\subsubsection{SCC1, taglio 0.005}
\label{scc1_0-005}
\begin{figure}
\centering

\begin{subfigure}[b]{1\textwidth}
\includegraphics[scale=.3]{./img/stampe/scc1/0.png}
\includegraphics[scale=.3]{./img/stampe/scc1/1.png}
\includegraphics[scale=.3]{./img/stampe/scc1/2.png}
\end{subfigure}

\begin{subfigure}[b]{1\textwidth}
\includegraphics[scale=.3]{./img/stampe/scc1/3.png}
\includegraphics[scale=.3]{./img/stampe/scc1/4.png}
\includegraphics[scale=.3]{./img/stampe/scc1/5.png}
\end{subfigure}

\begin{subfigure}[b]{1\textwidth}
\includegraphics[scale=.3]{./img/stampe/scc1/6.png}
\includegraphics[scale=.3]{./img/stampe/scc1/7.png}
\includegraphics[scale=.3]{./img/stampe/scc1/8.png}
\end{subfigure}
\begin{subfigure}[b]{1\textwidth}
\includegraphics[scale=.3]{./img/stampe/scc1/9.png}
\includegraphics[scale=.3]{./img/stampe/scc1/10.png}
\includegraphics[scale=.3]{./img/stampe/scc1/11.png}
\end{subfigure}
\caption{SCC1, Taglio 0.005, h 0-11. Da vedere da sinistra verso destra e dall'alto verso il basso}
\end{figure}
\begin{figure}
\begin{subfigure}[b]{1\textwidth}
\includegraphics[scale=.3]{./img/stampe/scc1/12.png}
\includegraphics[scale=.3]{./img/stampe/scc1/13.png}
\includegraphics[scale=.3]{./img/stampe/scc1/14.png}
\end{subfigure}
\begin{subfigure}[b]{1\textwidth}
\includegraphics[scale=.3]{./img/stampe/scc1/15.png}
\includegraphics[scale=.3]{./img/stampe/scc1/16.png}
\includegraphics[scale=.3]{./img/stampe/scc1/17.png}
\end{subfigure}
\begin{subfigure}[b]{1\textwidth}
\includegraphics[scale=.3]{./img/stampe/scc1/18.png}
\includegraphics[scale=.3]{./img/stampe/scc1/19.png}
\includegraphics[scale=.3]{./img/stampe/scc1/20.png}
\end{subfigure}
\begin{subfigure}[b]{1\textwidth}
\includegraphics[scale=.3]{./img/stampe/scc1/21.png}
\includegraphics[scale=.3]{./img/stampe/scc1/22.png}
\end{subfigure}
\caption{SCC1, Taglio 0.005, h 12-22; da vedersi da sinistra verso destra e dell'alto verso il basso.}
\end{figure}
In questo caso, le componenti sono per la maggior parte delle ore estremamente ampie, come si può vedere in seguito:
\begin{figure}
\begin{verbatim}
0:      6,
1:      2972,11,
2:      4,2,13,10,38,40,9,
3:      462,19,7,16,
4:      7,41,7,14,14,
5:      2563,2,18,
6:
7:      8010,
8:      6605,29,
9:      5677,5,8,5,
10:     5638,4,
11:     5492,4,5,3,13,6,
12:     5729,4,
13:     5692,14,11,
14:     5818,5,13,5,8,6,12,
15:     5784,5,12,3,5,
16:     5362,11,
17:     5118,9,7,15,14,5,
18:     5623,8,2,11,11,14,
19:     5951,5,5,10,
20:     6676,11,6,
21:     7449,12,
22:     7971,2,
\end{verbatim}
\caption{Nel listato, per ogni ora (alla sinistra), una lista delle dimensioni delle CFC trovate}
\label{list:scc1_0-005}
\end{figure}
Si noti in ~\ref{list:scc1_0-005} come la dimensione delle componenti può variare anche notevolmente a seconda del traffico presente in un'ora. Questo potrebbe essere dovuto al fatto che il taglio è estremamente basso, e pertanto agisce solamente su una parte minima degli archi. Nelle ore in cui il traffico è minore, anche le componenti hanno dimensioni notevolmente minori: probabilmente questo è dovuto al fatto che il numero di archi uscenti (fuori dall'orario lavorativo) è minore e con probabilità maggiore.
\subsection{SCC1, taglio 0.05}
In questo caso, invece, le dimensioni delle componenti fortemente connesse diminuiscono notevolmente. Si noti, a conferma dell'ipotesi fatta in \ref{scc1_0-005}, come le dimensioni delle CFC calino drasticamente nelle ore di maggior traffico, mentre le ore che hanno un traffico inferiore sono caratterizzate da CFC più ampie.
Si veda \ref{list:scc1_0-05}. Questo sembra indicare la necessità di effettuare tagli diversi per ore diverse, in modo da poter gestire diverse distribuzioni di traffico.
\begin{figure}
\begin{verbatim}
0:      4,7,2,2,6,
1:      383,2,4,7,7,6,6,9,5,9,2,
2:      6,4,6,11,5,4,4,2,8,8,
3:      5,6,5,11,9,
4:      6,4,8,6,6,
5:      8,4,3,5,8,3,7,2,3,
6:      718,4,2,10,9,3,6,3,7,3,6,7,3,
7:      115,6,2,2,3,2,2,2,3,4,
8:      2,3,
9:
10:     2,2,
11:     2,
12:     3,
13:     2,
14:
15:     4,
16:
17:     2,
18:     4,5,5,
19:
20:
21:     2,2,2,2,4,5,2,5,
22:     248,2,2,2,2,4,2,6,6,4,2,2,2,2,9,
\end{verbatim}
\caption{Da leggersi come in \ref{list:scc1_0-005}}
\label{list:scc1_0-05}
\end{figure}
Il risultato ottenuto, anche in questo caso non è stato considerato soddisfacente. Come esempio, mostriamo in \ref{list:scc1_0-05_1011cfc} le componenti fortemente connesse trovate alle ore 10 e alle ore 11. Oltre alla loro dimensione - certamente non eccessiva - si può vedere che le CFC in 10 scompaiono in 11.
Il dato confortante è che le CFC sono contigue geograficamente; ciò parebbe confermare l'ipotesi di una correlazione tra frequenza delle chiamate e vicinanza geografica.
\begin{figure}
\begin{subfigure}[b]{1\textwidth}
\begin{verbatim}
==== 10.cfc
{7249,7250}
{7724,7824}
\end{verbatim}
\begin{verbatim}
====11.cfc
{3659,3660}
\end{verbatim}
\end{subfigure}
\caption{Le CFC trovate alle ore 10 e alle ore 11 del 15 Novembre 2013}
\label{list:scc1_0-05_1011cfc}
\end{figure}
\subsection{SCC2, taglio 0.005}
Il comportamento evidenziato con questo tipo di visita, che, ricordiamo, seleziona prima i nodi con minore traffico e dalla stella uscenti quelli con minore probabilità, non danno risultati estremamente diversi.
%%todo: continuare.

\end{document}
