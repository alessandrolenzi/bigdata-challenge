\documentclass[10pt,a4paper]{article}
\usepackage[utf8]{inputenc}
\usepackage[italian]{babel}
\usepackage{amsmath}
\usepackage{amsfonts}
\usepackage{amssymb}
\usepackage{graphicx}
\usepackage{fullpage}
\author{Michele Carignani, Alessandro Lenzi}
\title{Analisi aggregata per ore}
\begin{document}

\maketitle

\section{Generazione dei grafi orari}

Per prima cosa i dati sono stati aggregati per ora. I dati originali del dataset \verb!MItoMI! sono nel formato:
\begin{verbatim}
timestamp \t SourceId \t DestId \t Stregth
\end{verbatim}
e sono stati suddivisi in 24 files (uno per ogni ora) e aggregati, per cui
ogni file contiene (al massimo\footnote{poichè certi nodi possono non avere chiamate in uscita
in una certa fascia oraria.}) un record per ogni nodo nel formato:
\begin{verbatim}
\SourceId \t DestId:Strength [\t DestId:Strength]
\end{verbatim}

A questo punto i pesi sugli archi (sopra chiamati \verb!Strength!) sono stati riscalati rispetto alla somma
dei valori della stella uscente di un nodo, ovvero:
$$ sumStrength_i = \sum_{j \in FS(i)} Strength_{ij} $$
$$ scaledStrength_{ij} = \frac{Strength_{ij}}{sumStrength_i} $$

\section{Ricerca delle componenti fortemente connesse}

\section{Risultati}

\subsection{Statistiche}

\subsection{Componenti Fortemente Connesse}

\begin{figure}[h]
\begin{center}
\begin{tabular}{|c|c|c|c|c|c|c|c|c|c|c|c|c|c|c|c|c|c|}
\hline 
Fascia Oraria & 0 & 1 & 2 & 3 & 4 & 5 & 6 & 7 & 8 & 9 & 10 &  \\ 
\hline 
Numero CFC & • & • \\ 
\hline 
Dimensioni CF & • & • \\ 
\hline 
\end{tabular} 
\caption{CFC scoperte con taglio AVG = 0.00291}
\end{center}
\end{figure}

\end{document}