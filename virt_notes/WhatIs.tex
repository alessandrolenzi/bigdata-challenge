A Virtual Machine (VM) is a machine decoupled from the hardware, meaning that to the eyes of its user, it does not necessarily correspond to the hardware but may mimick a different behaviour. 
Multiple Virtual Machines on the same physical host could share the underlying hardware.

The usual software development cycle, in which executables of an application are compiled for a certain operative system and for a certain architecture, makes portability difficult and thus requires for mechanisms to allow portability.
This is the main reason that led to the development of Virtual Machines concept, allowing a much higher degree of portability and flexibility.

Basically, a virtual machine is implemented through an additional software layer which is placed on an execution platform to give it the appearence of a different platform or even of multiple platforms.
In principle, all resources part of a computing system (including I/O devices, processors, memory and network resources) can be virtualized, making its state separate from a specific piece of physical hardware.

The easiest example of a VM are the multiprogrammed machine. In fact, the conventional multiprogrammed approach give to every process the illusion of having a complete machine dedicated (with its own memory, processor and so on), while the operating system timesharing mechanism,
along with hardware, make this possible. In this view, the OS provides a replicated VM for each of the applications executing in a certain system in a certain time instant.
\\
Several kinds of VM are available nowadays, and this technology, while considered promising for years, is now exploding due to the change in
the whole computing paradigm, in which computing systems are often shared between a huge users possibly not trusting each other, as it happens
in Clouds.
Virtual Machines, in fact, provide not just portability of software, but also interesting features in the field of \emph{security}, since they provide natural isolation. Other clear advantages are:
\begin{itemize}
\item Increased isolation between different users, otherwise possibly exploiting "holes" in the OS to damage others
\item Isolation wrt faults, software/versions, performances.
\item Better hardware exploitation, by reducing the number of servers and allowing access to multiple users; this allows also to save in power and cooling.
\item Better manageability in data centers, allowing applications to run seamless on different physical machines resources can be scheduled better and thus performance, power consumption and availability can be increased.
\item Improvements in software development cycles, by making development faster (consider, as a simple example, the advantages brought by the JVM to developers) 
\item Decreased complexity, by adding levels of abstraction to very complex computing system
\end{itemize}

In a modern computing system, a Virtual Machine can be seen as a mere abstraction level, that can be placed at higher or lower levels of the VM stack providing different virtualization techniques and different performances and features.
In VM, we can distinguish between \emph{guest}, corresponding to a software artifact running on the VM, and the \emph{host}, which is the underlying platform. The software which performs the abstraction defined below, is usually referred as \emph{Virtual Machine Monitor} (VMM) or, in some cases, \emph{Hypervisor}.
More formally, a virtual machine is defined as follows:
\begin{definition}
\textbf{Virtualization} is the construction of an isomorphism that maps a guest machine to an
existing host machine such that:
\begin{itemize}
\item for any possible guest state $R_i$ (collection of guest virtualization objects), there's a state $R'_i$ onto which $R_i$ is mapped using a certain function $V$ 
$$V(R_i) = R_i’$$
\item for every policy $P()$ (i.e. a command) causing a transition of guest state $R_i$ to state $R_j$, there
is a corresponding policy $P’()$ in the host that performs an equivalent modification of the host state
$$P’  V(R i )=V  P(R i )$$
\end{itemize}
\end{definition}

The main features required of virtual machines, according to Popek and Goldberg's 1974 paper %todo:insert ref
are:
\begin{enumerate}
\item \emph{Fidelity}: the software executing on a virtual machine executes identically to its execution in hardware, except for timing effects
\item \emph{Performance}: the majority of guest instructions are executed by the hardware without the intervention of the Virtual Machine Monitor
\item \emph{Safety}: the Virtual Machine Monitor manages all hardware resources.
\end{enumerate}