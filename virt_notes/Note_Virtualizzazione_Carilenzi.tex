%!TEX program = pdflatex
\documentclass[10pt]{article}
\usepackage{wrapfig}
\title{Virtualization Technologies}
\author{Michele Carignani, Alessandro Lenzi}
\newtheorem{theorem}{Theorem}[section]
\newtheorem{lemma}[theorem]{Lemma}
\newtheorem{proposition}[theorem]{Proposition}
\newtheorem{corollary}[theorem]{Corollary}

\newenvironment{proof}[1][Proof]{\begin{trivlist}
\item[\hskip \labelsep {\bfseries #1}]}{\end{trivlist}}
\newenvironment{definition}[1][Definition]{\begin{trivlist}
\item[\hskip \labelsep {\bfseries #1}]}{\end{trivlist}}
\newenvironment{example}[1][Example]{\begin{trivlist}
\item[\hskip \labelsep {\bfseries #1}]}{\end{trivlist}}
\newenvironment{remark}[1][Remark]{\begin{trivlist}
\item[\hskip \labelsep {\bfseries #1}]}{\end{trivlist}}

\newcommand{\qed}{\nobreak \ifvmode \relax \else
      \ifdim\lastskip<1.5em \hskip-\lastskip
      \hskip1.5em plus0em minus0.5em \fi \nobreak
      \vrule height0.75em width0.5em depth0.25em\fi}
\begin{document}

\maketitle
\tableofcontents
\section{What is Machine Virtualization}
A Virtual Machine (VM) is a machine decoupled from the hardware, meaning that to the eyes of its user, it does not necessarily correspond to the hardware but may mimick a different behaviour. 
Multiple Virtual Machines on the same physical host could share the underlying hardware.

The usual software development cycle, in which executables of an application are compiled for a certain operative system and for a certain architecture, makes portability difficult and thus requires for mechanisms to allow portability.
This is the main reason that led to the development of Virtual Machines concept, allowing a much higher degree of portability and flexibility.

Basically, a virtual machine is implemented through an additional software layer which is placed on an execution platform to give it the appearence of a different platform or even of multiple platforms.
In principle, all resources part of a computing system (including I/O devices, processors, memory and network resources) can be virtualized, making its state separate from a specific piece of physical hardware.

The easiest example of a VM are the multiprogrammed machine. In fact, the conventional multiprogrammed approach give to every process the illusion of having a complete machine dedicated (with its own memory, processor and so on), while the operating system timesharing mechanism,
along with hardware, make this possible. In this view, the OS provides a replicated VM for each of the applications executing in a certain system in a certain time instant.
\\
Several kinds of VM are available nowadays, and this technology, while considered promising for years, is now exploding due to the change in
the whole computing paradigm, in which computing systems are often shared between a huge users possibly not trusting each other, as it happens
in Clouds.
Virtual Machines, in fact, provide not just portability of software, but also interesting features in the field of \emph{security}, since they provide natural isolation. Other clear advantages are:
\begin{itemize}
\item Increased isolation between different users, otherwise possibly exploiting "holes" in the OS to damage others
\item Isolation wrt faults, software/versions, performances.
\item Better hardware exploitation, by reducing the number of servers and allowing access to multiple users; this allows also to save in power and cooling.
\item Better manageability in data centers, allowing applications to run seamless on different physical machines resources can be scheduled better and thus performance, power consumption and availability can be increased.
\item Improvements in software development cycles, by making development faster (consider, as a simple example, the advantages brought by the JVM to developers) 
\item Decreased complexity, by adding levels of abstraction to very complex computing system
\end{itemize}

In a modern computing system, a Virtual Machine can be seen as a mere abstraction level, that can be placed at higher or lower levels of the VM stack providing different virtualization techniques and different performances and features.
In VM, we can distinguish between \emph{guest}, corresponding to a software artifact running on the VM, and the \emph{host}, which is the underlying platform. The software which performs the abstraction defined below, is usually referred as \emph{Virtual Machine Monitor} (VMM) or, in some cases, \emph{Hypervisor}.
More formally, a virtual machine is defined as follows:
\begin{definition}
\textbf{Virtualization} is the construction of an isomorphism that maps a guest machine to an
existing host machine such that:
\begin{itemize}
\item for any possible guest state $R_i$ (collection of guest virtualization objects), there's a state $R'_i$ onto which $R_i$ is mapped using a certain function $V$ 
$$V(R_i) = R_i’$$
\item for every policy $P()$ (i.e. a command) causing a transition of guest state $R_i$ to state $R_j$, there
is a corresponding policy $P’()$ in the host that performs an equivalent modification of the host state
$$P’  V(R i )=V  P(R i )$$
\end{itemize}
\end{definition}

The main features required of virtual machines, according to Popek and Goldberg's 1974 paper %todo:insert ref
are:
\begin{enumerate}
\item \emph{Fidelity}: the software executing on a virtual machine executes identically to its execution in hardware, except for timing effects
\item \emph{Performance}: the majority of guest instructions are executed by the hardware without the intervention of the Virtual Machine Monitor
\item \emph{Safety}: the Virtual Machine Monitor manages all hardware resources.
\end{enumerate}
\subsection{What to virtualize: API,ABI,ISA}

Modern computer system is very complex: 
Hundreds of millions of transistors
Interconnected high-speed I/O devices
Networking infrastructures
Operating systems, libraries, applications
Graphics and networking software


To manage this complexity we use Levels of Abstractions which
Allows implementation details at lower levels of design to be ignored or simplified,
Each level is separated by well-defined interfaces, so
that the design of a higher level can be decoupled
from the lower levelsLayers of Abstraction


Abstraction: used to manage complexity typically defined in layers (VM i ), each layer has its own language (L i ) and data structures (R i )   lowest layers implemented in hardware, higher layers implemented in
software

A Machine: denotes the system on which software is executed.

Typical Layers are 
VM 4 : Applications 
VM 3 : Operating System
VM 2 : Assembler Machine
VM 1 : Firmware Machine
VM 0 : Hardware Machine

–  to an operating system this is generally the physical system
–  to an application program a machine is defined by the combination of hardware and OS-implemented abstractions
 

Abstraction layers have well defined interfaces
–  A processors instruction set defines such an interface: IA-32, IBM
PowerPC, ARM
•  Instruction Set Architecture (ISA)
–  defines hardware/software boundary
–  user ISA: portion of architecture visible to an application program
–  system ISA: portion of architecture visible to the supervisor software
(e.g., OS)
Applications
Operating
System
system ISA
user ISA
Execution
Hardware

ISA

Application Binary Interface (ABI): defines program interface to the hardware resources and services 
user ISA
system calls
• 
• 
• 
system instructions are not included in the ABI
user instructions allow program direct access to hardware
indirect interface for accessing shared system resources and services
implemented by the supervisor software
Application Programming Interface (API)
– 
– 
– 
defined in terms of a high-level language (HHL)
typically implemented as a system library and defined at the source level (for example libc which is linked
into programʼ’s address space)
specifies operations available by system which are implemented by the operating system or other system
software
Applications
API
Library
Operating
System
ABI
Execution
HardwareVirtualization

What is Virtualization
Linux
Linux
(devel)
XP
Vista
Virtual Machine Monitor
Hardware
MacOS

\subsection{A formal definition}



\subsection{Virtualization Properties}

•  Isolation
•  Encapsulation
•  InterpositionIsolation


\paragraph{Fault Isolation}
–  Fundamental property of virtualization
•  Software Isolation
–  Software versioning
–  DLL Hell
•  Performance Isolation
–  Accomplished through scheduling and resource
allocation

\paragraph{Encapsulation}
•  All VM state can be captured into a file
–  Operate on VM by operating on file
–  mv, cp, rm
•  Complexity
–  Proportional to virtual HW model
–  Independent of guest software configuration

\paragraph{Interposition}
•  All guest actions go through monitor
•  Monitor can inspect, modify, deny
operations
•  Examples:
–  Compression
–  Encryption
–  Profiling
–  Translation

\paragraph{Concrete Perspectives}
•  Process perspective: The system ABI defines the interface
between the process and machine
–  user-level hardware access: logical memory space, user-level
registers and instructions
–  OS mediated: Machine I/O or any shared resource or operations
requiring system privilege.
•  Operating system perspective: ISA defines the interface between
OS and machine
–  system is defined by the underlying machine
–  direct access to all resources
–  manage sharing
•  Virtual machine executes software (process or operating system)
in the same manner as target machine
–  Implemented with both hardware and software
–  VM resources may differ from that of the physical machine
–  Generally not necessary for VM to have equivalent performace

\paragraph{Where is the VM?}
Process virtual machine: supports an individual process
–  Emulates user-level instructions and operating system calls
–  Virtualizing software placed at the ABI layer
System Virtual Machines: emulates the target hardware ISA
–  guest and host environment may use the same ISA
Virtual Machines are implemented as combination of
–  Real hardware
–  Virtualizing software
Process
Virtualization Software
Operating System
ISA
Applications
Operating System
Virtualization Software
ISA


\paragraph{Terminology}

host environment: layers under the VM
guest environment: layers above the VM
runtime: virtualizing software in process VMs.
virtual machine monitor (VMM): virtualizing software
in system VMsVM Capabilities
Virtual machines can provide emulation, optimization
and replication
•  emulation: cross platform compatibility
•  optimization: by considering implementation specific
information
•  replication: making a single resource or platform appear as
many
Apps
Apps
OS
Apps
Apps
OS
OS
Apps
OS
OS
ISA ISA ISA
Emulation Replication CompositionProcess VM Examples
•  Same ISA
–  Multiprogrammed Systems
–  Binary Optimizers
•  Different ISA
–  Emulators \& Dynamic Binary Translators
–  Higher Level Language (HLL) VMsSystem VM Examples
•  Whole System
–  Same/Different ISA
–  Bare Hardware/Native/Bare Metal/Type 1
–  Hosted/Type 2
VM 1 VM 2
OS OS
VM 1 VM 2 OS OS Scheduler
HW
Drivers Scheduler
Scheduler
MMU
VMM
MMU
VMM MMU
OS HW
Drivers
Hardware Hardware
Native Virtualization Hosted Virtualization
•  Codesigned
–  innovative ISAs and/or hardware implementations for
improved performance, power efficiency, or both.


\paragraph{Taxonomy}

Process	VMs	
	Same ISA	
		Multiprogrammed	Systems	
		Dynamic	binary optimizers
	Different	 ISA	
		Dynamic	Translators
		HLL	 VMs	
 	 
System	VMs	
    Same ISA	
    	Classic-­‐System
    	Hosted VMs	
  	Different ISA VMs	
   		Whole-­‐System VMs	
   		
   		Codesigned VMs
  
\section{When}

–  First VM: IBM System/360 Model 40 VM [1965]Why Virtualize?
• First VM: IBM System/360 Model 40 VM [1965]

\section{Why Virtualization}

\paragraph{Consolidation}:
save resources by sharing them. Servers are under-utilized
because of over-provisioning approaches
•  Consolidate resources
–  Server consolidation
–  Client consolidation
–  reduce number of servers
–  reduce space, power and cooling
–  70-80% reduction numbers cited in industry
•  Client consolidation
–  developers: test multiple OS versions, distributed
application configurations on a single machine
–  end user: Windows on Linux, Windows on Mac
–  reduce physical desktop space, avoid managing
multiple physical computersImprove system management


\paragraph{System management} 
CLOUD
–  For both hardware and software
–  From the desktop to the data center
•  Data center management
–  VM portability and live migration a key enabler
–  automate resource scheduling across a pool of
servers
–  optimize for performance and/or power consumption
–  allocate resources for new applications on the fly
–  add/remove servers without application downtime
•  Desktop management
–  centralize management of desktop VM images
–  automate deployment and patching of desktop VMs
–  run desktop VMs on servers or on client machines
•  Industry-cited 10x increase in sysadmin
efficiencyImprove the software lifecycle



•  Improve the software lifecycle
–  Develop, debug, deploy and maintain
applications in virtual machines

•  Increase application availability
–  Fast, automated recoveryConsolidate resources


•  Develop, debug, deploy and maintain
applications in virtual machines
•  Power tool for software developers
–  record/replay application execution deterministically
–  trace application behavior online and offline
–  model distributed hardware for multi-tier applications
•  Application and OS flexibility
–  run any application or operating system
•  Virtual appliances
–  a complete, portable application execution
environmentIncrease application availability
•  Fast, automated recovery
–  automated failover/restart within a cluster
–  disaster recovery across sites
–  VM portability enables this to work reliably across
potentially different hardware configurations
•  Fault tolerance
–  hypervisor-based fault tolerance against hardware
failures [Bressoud and Schneider, SOSP 1995]
–  run two identical VMs on two different machines,
backup VM takes over if primary VM’s hardware
crashes
–  commercial prototypes beginning to emerge (2008)

Why?
• Consolidate resources
• Server consolidation
• Client consolidation
• Improve system management
• For both hardware and software
• From the desktop to the datacenter
• Improve software lifecycle
• Develop, debug, deploy and maintain applications in virtual
machines
• Increase application availability
• Fast, automated recovery

\section{How: Implementation techniques}

Virtualization Implementations
• Full Virtualization
- Software Based
- VMware and Microsoft
• Para Virtualization
- Cooperative virtualization
- Modified guest OS
- VMware, Xen
• Hardware-assisted Virtualization
- Unmodified guest OS
- VMware and Xen on virtualization-aware hardware platforms

\section{Memory virtualization}

Memory Virtualization

(based on Scott Devine slides by VMWare)


Traditional Address Spaces
0
4GB
Operating
System
Current Process
0
Virtual
Address Space
4GB
RAM
Frame
Buffer
Devices
ROM
Physical
Address SpaceTraditional Address Spaces
0
4GB
Process Virtual Address Space
Background Process
Background Process
Operating
Operating
System
System
0
4GB
Operating
System
Current Process
0
Virtual
Address Space
4GB
RAM
Frame
Buffer
Devices
ROM
Physical
Address SpaceMemory Management Unit (MMU)
•  Virtual Address to Physical Address
Translation
–  Works in fixed-sized pages
–  Page Protection
•  Translation Look-aside Buffer
–  TLB caches recently used Virtual to Physical
mappings
•  Control registers
–  Page Table location
–  Current ASID
–  Alignment checkingTraditional Address Translation (I)
1
Virtual Address
2
TLB
Physical AddressTraditional Address Translation (II)
1
4
5
TLB
Virtual Address
2
Physical Address
3
Process
Process
Page
Process
Page
Process
Table
Page
Table
Page
Table
TableTraditional Address Translation (III)
1
5
8
TLB
Virtual Address
6
2
Physical Address
7
Process
Process
Page
Process
Page
Process
Table
Page
Table
Page
Table
Table
4
Page
Fault
Handler
3Virtualized Address Spaces
0
4GB
Current Guest Process
Virtual
Address Spaces
Guest OS
0
4GB
Virtual RAM
Virtual
Frame
Buffer
Virtual
Devices
0
Virtual
ROM
Physical
Address Space
4GB
RAM
Frame
Buffer
Devices
ROM
Machine
Address SpaceVirtualized Address Translation:
TLB Emulation
1
6
9
TLB
Virtual Address
7
2
Machine Address
5
8
Emulated
TLB
page
fault
VMM
traps
Guest
Page
Table
hidden
page
fault
4
3
true
page
fault
Guest OS
page fault
handler
4
Physical
Page
TableIssues
•  Guest page table consistency
–  What happens when the guest changes an entry
in its page table?
–  What happens when the guest switches to a new
page table on a process context switch?
•  Performance
–  Guest context switches flush entire software TLB
–  Minimize hidden page faults
–  Aggressive flushing will cause flood of hpfs every
guest context switch
–  Keep one shadow page table per guest processVirtualized Address Translation:
Shadow Page Tables
Virtual
CR3
Guest
Guest
Guest
Page Table Page Table Page Table
Shadow Shadow Shadow
Real CR3
Page Table
Page Table
Page TableGuest Write to CR3
Virtual
CR3
Guest
Guest
Guest
Page Table Page Table Page Table
Shadow Shadow Shadow
Real CR3
Page Table
Page Table
Page TableGuest Write to CR3
Virtual
CR3
Guest
Guest
Guest
Page Table Page Table Page Table
Shadow Shadow Shadow
Real CR3
Page Table
Page Table
Page TableUndiscovered Guest Page Table
Virtual
CR3
Guest
Guest
Guest
Page Table Page Table Page Table
Shadow Shadow Shadow
Real CR3
Page Table
Page Table
Page Table
Guest
Page TableUndiscovered Guest Page Table
Virtual
CR3
Guest
Guest
Guest
Guest
Page Table Page Table Page Table Page Table
Shadow Shadow Shadow Shadow
Real CR3
Page Table
Page Table
Page Table
Page TableVirtualized Address Translation:
Shadow Page Table
1
6
9
TLB
Virtual Address
7
2
Machine Address
5
8
Shadow
Page
Table
page
fault
VMM
traps
Guest
Page
Table
hidden
page
fault
4
3
true
page
fault
Guest OS
page fault
handler
4
Physical
Page
TableIssues
•  Benefits
–  Handle page faults in same way as Emulated TLB
–  Fast guest context switching
•  Page Table Consistency
–  Guest may not need invalidate TLB on writes to
off-line page tables
–  Need to trace writes to shadow page tables to
invalidate entries
•  Memory Bloat
–  Caching guest page tables takes memory
–  Need to determine when guest has reused page
tablesHardware-assisted Virtualization:
Nested Page Tables
•  Nested paging uses an additional nested page table (nPT) to translate guest
physical addresses to system physical addresses
•  The gPT maps guest linear addresses to guest physical addresses. Nested page
tables (nPT) map guest physical addresses to system physical addresses.
•  Guest and nested page tables are set up by the guest and hypervisor respectively.
When a guest attempts to reference memory using a linear address and nested
paging is enabled, the page walker performs a 2-dimensional walk using the gPT
and nPT to translate the guest linear address to system physical address.
•  Nested paging removes the overheads associated with shadow paging. Unlike
shadow paging, once the nested pages are populated, the hypervisor does not
need to intercept and emulate guest’s modification of gPT.
•  However because nested paging introduces an additional level of translation, the
TLB miss cost could be larger.Reclaiming Memory: ballooning
inflate balloon
(+ pressure)
Guest OS
may page out
to virtual disk
balloon
Guest OS
guest OS manages memory
implicit cooperation
balloon
deflate balloon
(– pressure)
Guest OS
may page in
from virtual diskPage Sharing
•  Motivation
–  Multiple VMs running same OS, apps
–  Deduplicate redundant copies of code, data,
zeros
•  Transparent page sharing
–  Map multiple PPNs to single MPN copy-on-write
–  Pioneered by Disco [Bugnion et al. SOSP ’97],
but required guest OS hooks
•  VMware content-based sharing
–  General-purpose, no guest OS changes
–  Background activity saves memory over timePage Sharing: Scan Candidate PPN
011010
110101
010111
101100
VM 1
VM 2
hash page contents
...2bd806af
VM 3
hint frame
Machine
Memory
Hash:
VM:
PPN:
MPN:
...06af
3
43f8
123b
hash
table



\section{I/O VirtualizationType of devices}
•  Dedicated Devices
–  Monitor, keyboard, mouse
–  No virtualization required, but VMM routing because guest
OS runs in user mode
–  Interrupt handled by VM on activation by VMM
•  Partitioned Devices
–  Disks
–  VMM maintains a map of parameters and re-issues the
requests to physical devices
•  Shared Devices
–  Network adapter
–  VMM translates through a virtual device drivers
•  Spooled Devices
–  Printer
•  Nonexistent Devices
–  Comm networkPerforming I/O
Applications
system calls
Operating System
driver calls
I/O Drivers
I/O operations
HardwareVirtualizing I/O
•  I/O Operations level
–  I/O runs in privileged mode
–  Trap in user mode
–  Difficult to reverse engineer a complete I/O action
•  Device drivers level
– 
– 
– 
– 
Needs virtual device drivers
VMM intercepts calls to virtual device drivers
Must know guest OS device driver implementation
Real drivers needed for native VMMs
•  System calls level
–  Most efficient
–  Must know guest OS ABI to I/O and rewrite it taking
care of emulation of everything else not directly
related to I/O.

\end{document}