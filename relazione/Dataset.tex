\section{Dataset}
\label{thedataset}

The dataset on which the analysis has been developed is an aggregated log of mobile phone
calls and short text messages with their geographical source and destination points.

The dataset is composed of several files, each one containing information of one specific day
within the observation period of two months (november and december 2013).

%% todo: inserire grid, nodes, edges

In each file, a row describes the connection strength between two geographical areas in the area surrounding Milan (Italy) during a time period of 10 minutes within the day.

The concept of connection strength between two areas is not formally defined: it is a decimal value proportional to the number of calls and sms sent from one area to another one.

Geographic areas are identified by a logical a grid of $10.000$ squared areas 
(with area $10.000m^2$) 
% todo: controllare numero
identified by an integer number $i \in [0, 9999]$ where the
$ i / 100 $ and $ i \% 100 $ are respectively the  
x and y coordinates of the node.

Therefore the format of each line is:

\begin{verbatim}
timestamp \t sourceNode	\t 	destNode \t strength
\end{verbatim}

where timestamp is the time in milliseconds describing the first millisecond of the period described.
The strength is an obfuscated value which is proportional to the interaction between two logical squares.
To sum up, the dataset describes a number of directed weighted graphs over
the nodes of the geopgraphical grid, one for each 10-minutes long interval
in the 2 months observation period. The total size of the dataset is around 345GB.

%% todo: size of the dataset

\section{Data distributions analysis}
\label{ds_analysis}

