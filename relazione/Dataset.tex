\section{Dataset}
\label{thedataset}

\paragraph{The dataset} on which the analysis has been developed is an aggregated log of mobile phone
calls and short text messages with their geographical source and destination points.
The dataset is composed by several textual files, each one containing information of one specific day
within the observation period of two months (november and december 2013).

\paragraph{In each file,} a row describes the \textbf{connection strength between two geographical areas} in the Province of Milan (Italy) during a time period of 10 minutes within the day.
The definition, given by the provider of the dataset, of connection strength between two areas is: a decimal value proportional to the number of calls and sms sent from one area to another one.

\paragraph{The grid} Geographic areas are identified by a logical a grid of $10.000$ squared areas (with area $10.000m^2$)
% todo: controllare numero
identified by an integer number $i \in [0, 9999]$ where the
$ i / 100 $ and $ i \% 100 $ are respectively the  
x and y coordinates of the node.

To give an example, the dataset is compose of $62$ text files, named like \verb!2013-11-01.txt!, containing lines with the format:

\begin{verbatim}
timestamp \t sourceNode	\t 	destNode
\end{verbatim}

where \verb!timestamp! is the time in milliseconds describing the first millisecond of the period described, \verb!sourceNode! is an integer describing the source and \verb!destNode! is an integer identifying the destination areas of call and messages within the logical grid.

Summing up, the dataset describes several \footnote{precisely there are 8784 graphs, one for each 10-minutes interval within the observation period of 61 days.} \textbf{directed weighted graphs over
the nodes of the geopgraphical grid}.
Records where the values is zero are not listed. Each file has a size between
3 and 6 GBs and the total size of the dataset is around 450GB.

\section{Data distributions analysis}
\label{ds_analysis}


