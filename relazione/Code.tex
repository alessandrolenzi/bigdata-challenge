The final code of the project is made available on a public \href{https://github.com/michele-carignani/hadoop-markov-clustering}{GitHub repository}, \\ https://github.com/michele-carignani/hadoop-markov-clustering.
The code is organized in two packages, one for the aggregation \texttt{aggregation} and one for the Markov Clustering procedure \texttt{markov\_clustering}.

\paragraph{Aggregation} package is composed of the following files:
\begin{itemize}
\item \texttt{Aggregator.java} comprises the logic for performing aggregations over periods of interest and labelling them.
\item \texttt{AverageReducer.java} is the reducer used in the first job, calculating for each chunk of processed data the average strength revealed.
\item \texttt{FilterCombiner.java} is the combiner used after the map in the first job to shrink the data.
\item \texttt{FilterMapper.java} performs the filtering and labeling phase in the first job.
\item \texttt{IdentityMapper.java} is used in the second job
\item \texttt{ProbabilityReducer.java} evaluates the average probability graph for all the defined periods of interests
\item \texttt{TimeAggregatedGraphs.java} is the driver defining the composition of all components.
\end{itemize}

\paragraph{Markov Clustering} package is composed of two subpackages. In \texttt{test} there are some utilities to check if the rows are stochastic, and is no longer used in the final project but has been used during the development.
For the remaining part, we have:
\begin{itemize}
\item package \texttt{blockmultiplication}, with the following files
\begin{itemize}
	\item \texttt{Block.java} represents a matrix block, and is used for partitioning the matrix
	\item \texttt{BlockColumnMapper.java} is used to read the values of the second matrix in the multiplication
	\item \texttt{BlockMultiplier.java} handles the multiplication of a single block
	\item \texttt{BlockRowMapper.java} is used to read the values of the first matrix for the multiplication
	\item \texttt{BlockSumReducer.java} aggregates the partial matrix block products to produce a block
	\item \texttt{BlockWiseMatrixMultiplication.java} deals with the partitioning of the matrix and starts their multiplication
	\item \texttt{MatrixBlocks.java} represents a partition of matrix blocks to be calculated together
	\item \texttt{MatrixMultiplicationReducer.java} is the reducer used to calculate the product of two blocks
	\item \texttt{PartialSumMapper.java} aggregates the partial matrix block products to be summed in the following reducer
\end{itemize}
\item \texttt{Convergence.java} runs a convergency check job
\item \texttt{ConvergenceMapper.java} mapper used in the convergency job
\item \texttt{ConvergenceReducer.java} reducer used in the convergency job
\item \texttt{DisjointComponentVisit.java} calcualates the components, performs the aggregations and writes in output the images
\item \texttt{Driver.java} overall driver of the clustering computation; contains the convergency loop
\item \texttt{Inflation.java} runs an inflation job.
\item \texttt{InflationReducer.java} reducer used in the inflation phase
\item \texttt{MatrixColumnMapper.java} not used
\item \texttt{MatrixRecomposer.java} Used to recompose the matrix after the clustering
\item \texttt{MatrixRowMapper.java} aggregates values by row; used in the inflation job
\item \texttt{MatrixSplitter.java} used to divide the initial matrices into blocks
\item \texttt{SplitterMapper.java} map used in the splitting phase
\item \texttt{SplitterReducer.java} reduce used in the splitting phase
\item \texttt{RecomposerMapper.java} is used to recompose the matrix in the final step of the MCL algorithm
\end{itemize}
