\section{Model}
In our model, we started by considering what is a community. In our personal interpretation of this word,
we defined a community as a set of individuals with close relationship, meaning high probability to interact or to have someone in common with whom they interact.

Since we're dealing with calls, the idea is that a community can be identified by the frequency of calls between members belonging
to the same community.

The idea behind this approach is that, since we're searching for a community, we don't care about their size (i.e. the probability for an individual to be in a certain "community"), 
but only the probability that two geographical areas (and thus, in our model, their inhabitants) can be correlated by their calls.

The assumptions that we made are:
\begin{itemize}
\item People are related to some zones more strongly, meaning that an individual 
has an higher probability of staying in certain zones (for example, near his work place or near his home) than in random places
in the area taken into consideration. This allows to relate calls from certain geographical zones to some people.
\item People call more often other people belonging to their community.
\item Is not of major interest who is colling whom, since we only care about the existance of a "binding" between the two individuals.
\item A person may belong to several communities, however depending on the day and on the time of the call, the probability of communicating with a certain community changes.
As an example, during work hours it is more probable that calls will be addressed to people related to the working activity, while outside it will involve more probably family and friends.
\footnote{Similarly, Telco operators have been known to distinguish between business traffic and residential traffic} 
\end{itemize}

This automatically leads to a model in terms of probability of calls outgoing from a certain zone, thus, assuming $N_{u}$ as the total population considered, we assumed the strength $S_{i,j}$ connecting
zone $i$ and zone $j$ in our dataset to be
$$
S_{i,j}  = P\{call(i,j)\} P\{a\ caller\ is\ in\ i\} N_{u}
$$

As it is possible to see, using the strengths as an indicator "as they are", even though the $N_{u}$ factor will have no influence, 
zones with an higher population or with an higher concentration of calls could obfuscate the behaviour of less populated zones, justifying 
why the probability approach has been followed.

The probability of one call between grid $i$ and grid $j$ can be thus extimated as
$$
P_{i,j} = P\{call(i,j\} = \frac{S_{i,j}}{\sum_{j}{}{S_{i,j}}}
$$

We can think of the initial dataset as composed of several graphs, one for each 10 minutes slots, in which the strengths $S_{i,j}$ are seen as weights of the arcs connecting two zones $S_{i,j}$. 
However notice that using a single 10 minutes graph is not particularly significative, as noise could change significantly the results.

So, to get more realistic results and to deal with the very high level of noise in the dataset,
we decided to gather and average the strength of interaction between couples of nodes in several 10-minutes periods
considering a semantically meaningful aggregation in time (not alway contiguous, e.g. we aggregated all monday
mornings periods), in which we assumed that the behaviour of commmunities would be more or less the same.
\label{AVGPG}
We called these aggregations \emph{average probability graphs}, in which nodes are cells in the grid of our dataset, arcs represent interaction and their weight the average probability that such an interaction is established

To be more formal, let $E$ be the set of arcs of the initial dataset and $E_{id} \subseteq E$ be the set of arcs belonging to aggregation $id$ composed of $k$, then the weight $w_{id}(e)=\widehat{P_{i,j}}$ of an arc $e=(i,j)$ of the average graph for aggregation period $id$ will be 
$$
w_{id}(e) = \frac{\sum_{\substack{e' \in E_{id}}} w(e')}{k}
$$

